\documentclass[conference]{IEEEtran}
\IEEEoverridecommandlockouts
% The preceding line is only needed to identify funding in the first footnote. If that is unneeded, please comment it out.
%\usepackage[table]{xcolor}
\usepackage{paralist} % compactitem etc
\usepackage{booktabs} % For formal tables
\usepackage{numprint}
\usepackage{xspace}
%\usepackage{cite}
\usepackage{amsmath,amssymb,amsfonts}
\usepackage{algorithmic}
\usepackage{graphicx}
\usepackage{color}
\usepackage{textcomp}
\usepackage[normalem]{ulem}
\usepackage{soul}
\usepackage{multirow}
\usepackage{dblfloatfix}
\usepackage{listings}
\usepackage[utf8]{inputenc}
\usepackage[english]{babel}
\usepackage[noadjust]{cite}
\usepackage{caption}
\usepackage{subcaption}

\usepackage[table,xcdraw]{xcolor}
%\usepackage[noline,noend]{algorithm2e}
\usepackage[noline]{algorithm2e}


\newcommand{\ignore}[1]{ } % ignore blocks of text
\newcommand{\maketiny}[1]{{\tiny G: #1}} % ignore blocks of text
\newcommand{\ganesh}[1]{\todo[inline,size=\small, color=purple!40]{G: #1}}

\newcommand{\parlot}{\textsc{ParLoT}\xspace}
\newcommand{\pin}{\textsc{PIN}\xspace}
\newcommand{\callgrind}{\textsc{Callgrind}\xspace}
\newcommand{\parlotm}{\textsc{ParLoT(m)}\xspace}
\newcommand{\parlota}{\textsc{ParLoT(a)}\xspace}
\newcommand{\parlotnc}{\textsc{ParLoT-NC}\xspace}
\newcommand{\pininit}{\textsc{Pin-Init}\xspace}
%--\newcommand{\circleRtool}{circleRtool\circledR\xspace}

%%% Local Variables:
%%% mode: latex
%%% eval: (flyspell-mode 1)
%%% TeX-master: "root.tex"
%%% End:


\renewcommand{\algorithmcfname}{Procedure}


\makeatletter
\def\BState{\State\hskip-\ALG@thistlm}
\makeatother

\definecolor{codegreen}{rgb}{0,0.3,0}
\definecolor{codegray}{rgb}{0.5,0.5,0.5}
\definecolor{codepurple}{rgb}{0.58,0,0.82}
\definecolor{backcolour}{rgb}{0.98,0.98,0.98}
 
\lstdefinestyle{mystyle}{
    backgroundcolor=\color{backcolour},   
    commentstyle=\color{codegreen},
    keywordstyle=\color{magenta},
    numberstyle=\tiny\color{codegray},
    stringstyle=\color{codepurple},
    basicstyle=\ttfamily\scriptsize,,
    breakatwhitespace=false,         
    breaklines=true,                 
    captionpos=b,                    
    keepspaces=true,                 
    numbers=left,                    
    numbersep=4pt,                  
    showspaces=false,                
    showstringspaces=false,
    showtabs=false,                  
    tabsize=2,
    morekeywords={pragma, omp}
}
 
\lstset{style=mystyle}
\lstset{escapeinside={<@}{@>}}
 

\begin{document}
\bstctlcite{IEEEexample:BSTcontrol}
\title{DiffTrace: Efficient Whole-Program Trace Analysis and Diffing for Debugging
}



\author{\IEEEauthorblockN{Saeed Taheri}
\IEEEauthorblockA{\textit{School of Computing} \\
\textit{University of Utah}\\
Salt Lake City, Utah, USA \\
staheri@cs.utah.edu}
\and
\IEEEauthorblockN{Ian Briggs}
\IEEEauthorblockA{\textit{School of Computing} \\
\textit{University of Utah}\\
Salt Lake City, Utah, USA \\
ian.briggs@gmail.com}
\and
\IEEEauthorblockN{Martin Burtscher}
\IEEEauthorblockA{\textit{Department of Computer Science} \\
\textit{Texas State University}\\
San Marcos, Texas, USA \\
burtscher@txstate.edu}
\and
\IEEEauthorblockN{Ganesh Gopalakrishnan}
\IEEEauthorblockA{\textit{School of Computing} \\
\textit{University of Utah}\\
Salt Lake City, Utah, USA \\
ganesh@cs.utah.edu}
}


\IEEEoverridecommandlockouts
\IEEEpubid{\makebox[\columnwidth]{978-1-7281-4734-5/19/\$31.00~\copyright2019 IEEE \hfill} \hspace{\columnsep}\makebox[\columnwidth]{ }}

\maketitle

\IEEEpubidadjcol

\begin{abstract}
\label{abs}


\end{abstract}

\begin{IEEEkeywords}
Whole-program tracing, HPC debugging, trace diffing, nested loop recognition, formal concept analysis
\end{IEEEkeywords}


\section{Introduction}
\label{sec:intro}
\input{intro.tex}


\section{DiffTrace Overview}
\label{sec:overview}
\input{overview.tex}

\section{ Algorithms Underlying DiffTrace}
\label{sec:algo}
\input{algorithms.tex}

\section{Case Study: ILCS}
\label{sec:ilcs-case-study}
\input{ilcs.tex}
%\input{tabs/newResults.tex}

\section{LULESH2 Examples}
\label{sec:lulesh}
\input{lulesh.tex}

\section{Related Work}
\label{sec:related}
\input{relatedwork.tex}

\section{Discussions \& Future Work}
\label{sec:discussion}


\noindent{\bf Acknowledgements:\/} Supported in part by
NSF awards XXXXXX.



\bibliographystyle{IEEEtran}
\bibliography{bibs}


%\appendix
%\section{Additional Material}
%\input{sup.tex}

\end{document}
